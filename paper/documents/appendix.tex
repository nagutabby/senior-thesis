
\section*{付録}\addcontentsline{toc}{section}{付録}

研究遂行に際して開発したスクリプトや,解決に苦労した事項,作成したソースコード等,本編で述べるまでは無いが,後輩や未来の自分のために残しておきたい内容をここに書く.
その際必ず本文中で参照する.たとえば「\autoref{src:constructor-destructor}ではC++におけるコンストラクタとデストラクタの宣言と実装のための書式を示している.」のようにきちんと参照して述べる.

参考文献と付録は下限であるページ数25以上の条件にカウントしない.



\begin{figure}[!tb]
	\begin{lstlisting}[caption=コンストラクタとデストラクタの書式, label=src:constructor-destructor, basicstyle=\ttfamily\scriptsize,escapechar=@]
class クラス名{
  private:
    プライベート変数; // 他のクラスからはアクセス不可
    プライベート関数(); // 他のクラスからはアクセス不可
	public:
	  クラス名(); //デフォルトコンストラクタ
	  クラス名(引数の型); //コンストラクタ1
	  クラス名(引数の型, 引数の型); //コンストラクタ2
	  ~クラス名();//デストラクタ
};

クラス名::クラス名(){
	// デフォルトコンストラクタの実装
}

クラス名::クラス名(引数の型 引数){//コンストラクタ1
	// コンストラクタの実装
}

クラス名::クラス名(引数の型 引数, 引数の型 引数){//コンストラクタ2
	// コンストラクタの実装
}

クラス名::~クラス名(){
	// デストラクタの実装
}

int main(){
	クラス名 変数名;//デフォルトコンストラクタの呼び出し
	クラス名 変数名(引数);//コンストラクタ1の呼び出し
	クラス名 変数名(引数, 引数);//コンストラクタ2の呼び出し
	クラス名 変数名[10];//デフォルトコンストラクタの呼び出し(10回)
	return 0;//デストラクタがreturnの直前にそれぞれの変数で呼び出される(合計13回)
}
	\end{lstlisting}
\end{figure}
