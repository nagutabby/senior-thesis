\section{まとめ}\label{sec:まとめ}

ソースコードは\autoref{src:fizzbuzz.cpp}のように掲載する.\LaTeX{}のソースコードに貼り付けるのではなく,ファイルのパスを指定して直接的に読み込むこともできる.

\begin{figure}[!tb]
	\begin{lstlisting}[caption=ソースコードの掲載例, label=src:fizzbuzz.cpp, basicstyle=\ttfamily\scriptsize,escapechar=@]
#include <iostream>
#include <numeric>
#include <map>

int main() {
  const std::map<const int, std::string> m{
    {3, "Fizz"},
    {5, "Buzz"},
    {15, "FizzBuzz"}
  };
  for(int i{1}; i < 100; ++i){
    const int n{std::gcd(i, 15)};
    if (m.contains(n)){ // C++20の機能
      std::cout << m.at(n) << '\n';
    }else{
      std::cout << i << '\n';
    }
  }
  return 0;
}
	\end{lstlisting}
\end{figure}
