\section{初めに}
\label{section:初めに}
通信インフラの整備状況は地域間で大きく異なる。例えば、モバイルネットワークの世代別のシェアを見ると、世界では2022年時点で4Gネットワークが88\%を占めるが、アフリカでの4Gネットワークの普及率は50\%、アラブ諸国では76\%となっている~\cite{ITU2022FactsAndFigures}。所得別で見ると低中所得以上の人々の間では比較的4Gネットワークの普及率が高いが、低所得の人々の間の普及率は34\%と低い値を示している。その他には、都市と地方の間にも同様の格差が残っていることが指摘されている。これらのデータは何を示唆しているのだろうか?
\subsection{背景}
\label{subsection:背景}
\subsubsection{モバイルネットワークの普及率の格差をもたらす要因}
\label{subsubsection:モバイルネットワークの普及率の格差をもたらす要因}
通信インフラへの投資が、モバイル通信システムの新しい世代の普及率に影響を及ぼしていることが指摘されている~\cite{Forge2020FormingA5GStrategyForDevelopingCountries}。通信インフラはモバイル通信サービスやオンラインサービスの基礎であるため、国々の間で激しい競争が行われているが、経済規模の違いから発展途上国や地方自治体は投資される側において不利である。膨大な予算を確保しやすい先進国や中央自治体の方がより魅力的な投資政策を立てられるためである。

他方で、通信システムを整備する組織の1つであるMNO(Mobile Network Operator)は、既にある通信設備の維持をしながら新しいモバイルネットワーク技術に多額の資金を投資する必要があるため、ある程度のリスクを抱えており、政府からの支援によって投資への積極性が左右されやすい。最近ではオンラインサービスを運営するMNOも登場しており、市場規模が拡大するにつれて明確な投資判断を下すために考慮するべき要素がますます増えている。

モバイル通信システムの世代が新しくなるにつれ、モバイル通信技術を導入するための費用が増加する傾向にある。例えば、モバイル通信システムの世代が新しいほど使用する周波数帯が高くなる傾向にあるが、これは1つの基地局が対応できる通信範囲が狭くなることを意味する。これにより、特定のカバレッジを確保する場合にかかる基地局などの設備費用が増加する。

さらに、高速な通信規格においては、5Gの要件であるeMBB(enhanced Mobile Broadband)、mMTC(massive Machine Type Communication)、uRLLC(Ultra-Reliable and Low Latency Communications)のような厳しい基準が定められており、開発に求められる技術が複雑化している。これは人件費の増加を招くため、モバイル通信技術を導入する際の1つの障壁となっている。

新しいモバイル通信技術はより高速なネットワークを提供するため、より多くの動画コンテンツに対応し、新しい市場や需要を産出する可能性があるが、その一方で現在の市場やエコシステムを大きく変える可能性もある。発展途上国や地方自治体は先進国や中央自治体に比べて受け取る投資額が少ないため、需要がどのくらいあるのかが明確ではない、新しいモバイル通信技術の開発や導入に消極的である。
\subsubsection{PWAの概要}
\label{subsubsection:PWAの概要}
通信技術の開発や通信インフラの整備状況は様々な要因の影響を受ける。これらの要因により十分な通信環境が確保できない地域に対してもユーザビリティーが高いサービスを提供することが必要である。特に、モバイル端末は安価である一方で、利用時には移動体通信が行われるため通信環境が変化しやすく、一定のユーザビリティーを確保することが難しい。このような問題を解決するために提案されている技術がPWA(Progressive Web Apps)である。

PWAはモバイルネイティブアプリに代表される、プラットフォーム固有のアプリのユーザビリティーを提供するWebの技術である。通常のWebアプリでは柔軟なキャッシュ管理ができないため、Webページの読み込み速度をキャッシュレベルで改善するための仕組みが複雑である。また、常にオンラインでアクセスする必要があるため、通信環境の地理的な制約を受けることがある。PWAを活用することでこれらの問題を解決できる。PWAを構成する要素をいくつか示す。

Web Application ManifestはWebアプリの情報を提供するJSONテキストファイルである。プラットフォーム固有のアプリが持っている、名前、説明、アイコン画像などのメタデータを記述する。必要な情報を記述することで、インストール時や起動時のUIを制御することもできる。開発者はWeb Application Manifestを使用することでユーザーにアプリの正確な情報を提示できるようになる。

Service WorkerはWeb Workerと呼ばれる仕組みの1つで、Webアプリ間、ブラウザー間、ネットワーク間でプロキシーサーバーとして動作する。オリジンサーバーへのネットワークリクエストを遮断し、ネットワークの利用状況に応じてオリジンサーバーの代わりにWebブラウザに対してネットワークレスポンスを返却することで、オフラインの際にもWebアプリを使用できるようにする。

PWAはWebアプリであるため、モバイル端末の一部のAPIにアクセスするためにはWeb APIを使用する必要がある。例えば、カメラ、マイク、位置情報、通知、端末の向きが挙げられる。Service Workerを制御する際にもWeb APIを使用する。一部のWeb APIは安全性が高いプロトコルであるHTTPSでのみ動作し、Web APIの対応状況はWebブラウザー間で異なることから、現在の対応状況や懸念点を考慮した上で、使用するものを決めることが重要である。PWAにおいて使用することが多いWeb APIや、Web APIの懸念点は後述する。

アプリは通常、コンテンツとそれを格納するUIからなる。App Shellはコンテンツを格納するUI(シェル)をコンテンツとは別にキャッシュすることで、アプリをオフライン環境で快適に動作させるためのアーキテクチャである。UIをコンテンツから分離することで読み込みが高速になり、3Gなどの低速な通信環境での読み込みの遅延を改善できる。

App Shellに基づいてキャッシュされる可能性があるネットワークリクエストはService Workerで制御される。また、Web APIを使用することで、シェルに加えてホーム画面への追加やプッシュ通知などの他の機能を追加できる。App Shellを採用することで、Webアプリで得られるユーザビリティーを、プラットフォーム固有のアプリで得られるユーザビリティーに近づけられる。
\subsubsection{位置情報市場の拡大とモバイルデータトラフィックの増加}
\label{subsubsection:位置情報市場の拡大とモバイルデータトラフィックの増加}
屋外の位置情報に関連する市場の規模は、国内では2025年度までに約1,900億円に拡大すると予想されている~\cite{MIC2023InformationStatistics}。位置情報は、地図、カーナビゲーション、マーケティング、タクシーの配車、ゲーム、家族や友人と位置情報を共有するアプリなどで活用されている。これらのサービスによって位置情報技術の認知度がより向上し、その認知度の高まりが位置情報市場の拡大を支えていると考えられる。

世界のモバイルデータトラフィックは2023年第2四半期から第3四半期にかけて前四半期比で7\%増加している~\cite{EricssonNovember2023MobilityReportDataAndForecasts}。モバイル通信システムの世代の移行が進んでいることや、画像、音声、動画などのコンテンツのアップリンクやダウンリンクのデータトラフィックが増加していることがこの要因である。動画コンテンツの視聴の増加が契約当たりの平均データ量の増加に関連していることから、配信されるコンテンツの種類の変化がモバイルデータトラフィックの増加に大きい影響を及ぼしていると考えられる。

このような背景を踏まえ、PWAの活用方法を検討する題材として観光案内を取り上げることにした。観光案内アプリは位置情報を使用する機会が多く、画像、音声、動画といったデータ量が大きいコンテンツを配信することが多いという特徴がある。

\subsection{目的}
\label{subsection:目的}
現在配信されている観光案内アプリに実装されている機能を調査し、観光案内アプリにPWAを導入する際の機能面の課題を明らかにする。関連研究で行われていたアプリのパフォーマンスの評価方法の問題点を指摘し、アプリのパフォーマンスをより正確に計測するための方法を示す。その方法を用いて、Service Workerのキャッシュ戦略を活用した場合の観光案内アプリのパフォーマンスを計測し、その戦略がどの程度効果的であるのかを示す。新たに提案したパフォーマンスの計測方法を用いてパフォーマンスを評価した場合に生じる問題点を指摘し、どのような改善が求められるのかを示す。
\subsection{この論文の構成}
\label{subsection:この論文の構成}
~\autoref{section:関連研究}では、PWAの性質、パフォーマンス、役割を論じている研究を紹介し、関連研究でまだ明らかにされていないことや、関連研究の課題を説明する~\autoref{section:研究内容}では、PWAを用いた観光案内アプリのパフォーマンスを評価する方法を述べる。~\autoref{section:評価}では、PWAを用いた観光案内アプリでの、Web APIの有用性とService Workerのパフォーマンスを評価する。~\autoref{section:まとめ}では、観光案内アプリにおけるPWAの有用性を評価し、観光案内アプリにPWAを導入する上での課題を考察する。その後にこの研究の目的を振り返り、研究で明らかになったことや今後さらなる研究が必要なことをまとめる。Service Workerのパフォーマンスを計測する際に使用したプログラムとそのプログラムの実行環境を付録に示す。