\section{はじめに}\label{sec:はじめに}

\subsection{学位論文の書き方}\label{subsec:学位論文の書き方}

提出する論文のページ総数は100ページ以下とする.
なお,表紙や目次,参考文献,付録は除いて\footnote{この例では,\autoref{sec:はじめに}から\autoref{sec:まとめ}に記載される,実際の論文の内容を指す.}25ページ以上とする.
図表の数に制限は設けないが,図表は必ず本文中で参照すること.本文で述べない図表は掲載しない.


必ず以下の手順で卒業論文を作成する.

\begin{enumerate}
  \item 課題と目的を明確にする・関連研究調査を行い研究計画を立てる
  \item 研究成果を得る
  \item 研究成果が目的達成または課題解決に役立っていることを多角的に評価および要確認する
  \item 読者が内容を理解するために必要な労力を最小化するよう心がけて論文の構成を考える(第1段階) \\ (例)
  \begin{itemize}
    \item はじめに
    \item 関連研究
    \item 研究内容
    \item 実装システム
    \item 評価
    \item まとめ
  \end{itemize}
  \item 読者が内容を理解するために必要な労力を最小化するよう心がけて論文の構成を考える(第2段階) \\ (例)
  \begin{itemize}
    \item はじめに
    \begin{itemize}
      \item 背景
      \item 目的
      \item 本論文の構成
    \end{itemize}
  \end{itemize}
    \item 読者が内容を理解するために必要な労力を最小化するよう心がけて論文の構成を考える(第$n$段階\footnote{図や表といったデータはこの段階で貼り付けておき,どの段落でどの図表を参照するか明記する.}) \\ (例)
    \begin{itemize}
      \item 背景
      \begin{itemize}
       \item 情報端末の普及
       \item 小型デバイスの登場
       \item ウェアラブルデバイスの登場
       \item プライバシーの問題
      \end{itemize}
    \end{itemize}
  \item 段落レベルまで構成を考えてから一週間寝かせる.
  \item 一週間後に自分でじっくりと確認・修正・他者の意見を参考にする
  \item 段落を文章で埋めつつ他者の意見を参考にして修正
  \item 完成
\end{enumerate}

\subsection{背景}\label{subsec:背景}

いろいろなことが研究されているが,
これこれについては未解決である.
そこで,これこれを作ることによって,
それそれがうまくできる可能性がある.
検証のためNNNNの開発と評価を行う.

% 1行開けると新たな段落となる

箇条書は itemize を使う.
\begin{itemize}
\item なまむぎ
\item なまごめ
\item なまたまご
\item なまはげ
\end{itemize}

\begin{table}[bt]\centering
  \caption{合図のタイミング\label{tab:signal}}
\begin{tabular}{|l|c|}\hline
種類 & タイミング              \\ \hline\hline
AU $ \rightarrow $ CU & ID     \\ \hline
AU $ \rightarrow $ BU & EX      \\ \hline
BU $ \rightarrow $ AU,BU & MEM \\\hline
BU $ \rightarrow $ AU,CU & ID \\ \hline
\end{tabular}
\end{table}

\subsection{構成}\label{subsec:構成}

本論文の構成を以下に示す.
まず\autoref{sec:はじめに}で研究の背景と目的,および本論文の構成について述べる.
続く\autoref{sec:関連研究}では〇〇する.
とくに\autoref{subsec:前提となる技術}で〇〇について述べる.
\autoref{sec:研究内容}で〇〇を提案・実装し,
\autoref{sec:評価}で〇〇を評価する.
最後に\autoref{sec:まとめ}で結論を述べる.






