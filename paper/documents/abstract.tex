% 500文字程度で概要を述べる
% 背景、目的、方法、結果、結果がどのような意味を持つのかを説明する文を含める
通信インフラや財政状況の地域間の格差により、均一なユーザー体験の提供を実現することが困難である。安価な電子機器であるモバイル端末の需要が大きいため、モバイルファーストなユーザー体験を実現することが重要である。モバイル端末に対して安定したユーザー体験を提供するための手段として、PWA(Progressive Web Apps)が注目されている。初めに、モバイルネイティブな観光案内アプリを比較対象として、PWAにおいてWeb標準のAPIを利用した場合のユーザー体験の変化について考察した。次に、観光名所を地図にマッピングするWebアプリにPWAを組み込み、Lighthouseとネットワークスロットリングを使用してPWAのパフォーマンスを調査した。その結果、モバイルネイティブな観光案内アプリに実装されている機能の多くはPWAを利用したアプリにおいても実装可能であり、キャッシュ管理を行うことでパフォーマンスの低下を抑えられることが示された。これらの結果から、観光案内アプリにおいてPWAを利用した場合は、通信インフラや財政的な制約を受ける環境においても安定したユーザー体験を提供できると考えられる。