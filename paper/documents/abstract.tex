% 500文字程度で概要を述べる
% 背景、目的、方法、結果、結果がどのような意味を持つのかを説明する文を含める
地域や所得によって十分な通信環境を確保できない人々が依然として存在するため、高いユーザビリティーを実現することが難しい。安価な電子機器であるモバイル端末の需要が大きいため、モバイル端末のユーザビリティーを向上させることが重要である。これを実現するための手段としてPWA(Progressive Web Apps)が注目されている。まず、モバイルネイティブの観光案内アプリを比較対象として、PWAにおいてWeb標準のAPIを利用した場合のユーザビリティーの変化を調査した。次に、観光スポットを地図にマッピングするWebアプリにPWAを組み込み、Lighthouseとネットワークスロットリングを用いてPWAのパフォーマンスを測定した。その結果、モバイルネイティブの観光案内アプリにおいて実装されている機能の多くはPWAで構成されているアプリでも実装可能であり、キャッシュ管理によってパフォーマンスの低下を抑えられることが分かった。これらの結果から、通信インフラや金銭的な制約を受ける環境においても、観光案内アプリでPWAを利用することで、高いユーザビリティーを実現できることが示された。