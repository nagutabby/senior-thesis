\section{関連研究}\label{section:関連研究}
関連研究としては、PWAを構成する要素の性質を調査したもの、PWAとモバイルネイティブアプリのパフォーマンスの比較によってPWAの有用性を評価したもの、PWAの開発方法に着目し、クロスプラットフォーム開発における役割を整理したものなどがある。それぞれの概要を説明し、関連研究における課題を整理する。
\subsection{PWAの構成要素とその性質}\label{subsection:PWAの構成要素とその性質}
PWAの構成要素にはApp Shell、Service Worker、Web Application Manifestがあり、それぞれの性質は以下の通りである~\cite{Tandel2018ProgressiveWebApps}。
\subsubsection{App Shellの性質}\label{subsubsection:App Shellの性質}
\begin{itemize}
    \item パフォーマンスが高い
    \begin{itemize}
      \item キャッシュされたコンテンツを利用することで高速にページを読み込めるため、通常のWebアプリと比べてパフォーマンスが高い。これはアプリに再びアクセスした際にページが即座に読み込まれることを意味する。
    \end{itemize}
    \item ネイティブアプリに近い操作性を実現
    \begin{itemize}
        \item Cache APIを利用することでオフライン環境でも動作する。PWAが考案される前は、ネイティブアプリのみがオフラインアクセスをサポートしていた。PWAによってネイティブアプリに近い操作性を実現するWebアプリが登場したことで、Webアプリの多様化が進んだ。
    \end{itemize}
    \item データの使用効率が高い
    \begin{itemize}
        \item UIをキャッシュするためページ遷移を行う際のデータ使用量が削減される。これによってデータの使用効率が高くなり、より少ないデータ使用量でページをレンダリングできるようになる。
    \end{itemize}
\end{itemize}
\subsubsection{Service Workerの性質}\label{subsubsection:Service Workerの性質}
\begin{itemize}
    \item オフラインアクセスを提供
    \begin{itemize}
        \item キャッシュしたネットワークレスポンスを使用してオフラインアクセスを提供できる。通信が不安定な環境でもアプリを利用できるようになるため、ユーザー体験が向上し、ユーザーに新たな価値を提供できるようになる、
    \end{itemize}
    \item プッシュ通知を提供
    \begin{itemize}
        \item ユーザーのモバイル端末やデスクトップ端末に直接送信されるメッセージであるプッシュ通知を提供できる。プッシュ通知を効果的に活用することで、アプリの利用を促進してコンバージョン率を向上させたり、行動パターンをより詳細に分析したりできるようになる。
    \end{itemize}
    \item バックグラウンドでコンテンツを更新
    \begin{itemize}
        \item Periodic Background Synchronization APIを用いることでバックグラウンドでコンテンツを更新できる。コンテンツを同期する際にユーザーの直接的な操作が不要になり、ユーザーに常に最新の情報を提供できる。
    \end{itemize}
    \item 柔軟なキャッシュ制御
    \begin{itemize}
        \item 一部のコンテンツのみをキャッシュしたり、一定の時間間隔でキャッシュを更新したりできるため、Cache Storageの使用量を削減してアプリのサイズを小さくしたり、古いコンテンツがキャッシュされたままになるのを回避できる。
    \end{itemize}
\end{itemize}
\subsubsection{Web Application Manifestの性質}\label{subsubsection:Web Application Manifestの性質}
\begin{itemize}
    \item アプリに関する情報を提供
    \begin{itemize}
        \item アプリの名前、説明、作者、アイコンのパスなどの情報を提供できる。これらのメタデータをWebアプリに追加することで、Webアプリを特定のプラットフォームに適合させられる。
    \end{itemize}
\end{itemize}
\subsubsection{PWAの構成要素とその性質に関する研究の課題}\label{subsubsection:PWAの構成要素とその性質に関する研究の課題}
実際のアプリでは、PWAが理想とする振る舞いを必ずしも実現できるとは限らない。例えば、バックグラウンドでのコンテンツの更新をサポートしている主なWebブラウザーは2023年11月時点でGoogle ChromeとMicrosoft Edgeのみであり、バックグラウンド同期に関する懸念も表明されている(詳しくは後述する)。また、Webブラウザー間でPWAの実装が異なるため、それを考慮する必要もある。
\subsection{パフォーマンスの比較}\label{subsection:パフォーマンスの比較}
Redditから画像とテキストを取得して表示する機能を、PWAとAndroidのネイティブアプリに実装し、アプリの最初のアクティビティが起動するまでの時間や、アプリのアイコンをタップしてからツールバーがレンダリングされるまでの時間を計測した研究がある~\cite{Andreas2018ProgressiveWebApps}。アプリの最初のアクティビティが起動するまでの時間を計測する際は、Android Debug Bridgeのam\_activity\_launch\_timeコマンドを使用しており、アプリのアイコンをタップしてからツールバーがレンダリングされるための時間を計測する際は、それぞれのアプリで10回操作を行い、その時間を手動で計測して平均時間を計算している。
\subsubsection{パフォーマンスの比較に関する研究の課題}\label{subsubsection:パフォーマンスの比較に関する研究の課題}
アプリのパフォーマンスは複合的なものであり、アプリの起動が始まるまでの時間やレンダリング完了までの時間に加えて、アプリのナビゲーションのしやすさも検証するべきである。また、さらに正確なパフォーマンスを計測するためにはネットワークのスループット、端末の幅と高さ、端末の性能も考慮する必要がある。関連研究ではいずれも考慮されていないため、パフォーマンスを計測する方法を改善する必要がある。
\subsection{クロスプラットフォーム開発における役割}\label{subsection:クロスプラットフォーム開発における役割}
現在のクロスプラットフォームフレームワークはプラットフォーム間で技術を統一できない~\cite{Majchrzak2018ProgressiveWebApps}。関連研究では、この問題を解決するための方法の1つとしてPWAを挙げている。PWAは単一のコードで複数のプラットフォームに対応できる点で他のクロスプラットフォームフレームワークとは異なる。PWAのようなクロスプラットフォームフレームワークの登場により学習工数やコストが削減され、市場投入までの時間が短縮される。
\subsubsection{クロスプラットフォーム開発における役割に関する研究の課題}\label{subsubsection:クロスプラットフォーム開発における役割に関する研究の課題}
プラットフォームごとにPWAのサポート状況が異なるため、PWAのWebブラウザー間の互換性を保つことが難しい。PWAの活用状況によってはPollyfillの導入が必要になる場合があり、それによってJavaScriptのサイズが大きくなり、クロスプラットフォーム開発の優位性が損なわれるかもしれない。