%\documentclass[lualatex,twoside,a4j]{ltjsarticle}
%\documentclass[oneside,a4j]{ltjsarticle}% 片面印刷用
\documentclass[twoside,a4j]{ltjsarticle}% 両面印刷用


\newcommand{\論文題目}{論文のタイトルをここに書く\\ 長い場合は二行目に渡っても良い}%% 改行の場所は \\ を使う
\newcommand{\英文題目}{A Study of Hoge Hoge Hoge Hoge}
\newcommand{\学籍番号}{1234567}
\newcommand{\クラス名列番号}{4EP0-00}
\newcommand{\氏名}{工大 次郎}    % 名字と名前の間に半角スペースを入れる

%%% 以下は固定
\newcommand{\年度}{5}
\newcommand{\提出日}{\和暦{}\today}
\newcommand{\指導教員}{坂本 真仁 講師}
\usepackage{sbase}

\begin{document}

\pagestyle{empty}
\vspace{10cm}
{\Large\bfseries
\noindent
令和\年度{}年度\ プロジェクトデザイン$\textrm{I\hspace{-.15em}I\hspace{-.15em}I}$ プロジェクトレポート\\
金沢工業大学\ 工学部\ 情報工学科\\

\vspace{4cm}
{\Huge\bfseries
\begin{center}
\論文題目
\end{center}
}
{\bfseries
\begin{center}
\英文題目
\end{center}
}

\vspace{6cm}
\begin{flushright}
\begin{tabular}{rl}
  学\hfill{}籍\hfill{}番\hfill{}号:
  & \学籍番号  \\
  ク\hfill{}ラ\hfill{}ス\hfill{}名\hfill{}列\hfill{}番\hfill{}号:
  & \クラス名列番号  \\
  氏\hfill{}名:
  & \氏名      \\
  提\hfill{}出\hfill{}日:
  & \提出日    \\
  指\hfill{}導\hfill{}教\hfill{}員:
  & \指導教員  \\
\end{tabular}
\end{flushright}
}

\clearpage
%%%%% 両面印刷の場合に表紙の裏を白紙にする
%%%%% 差し替えたときに目次を表紙の裏にしないため
%\thispagestyle{empty}
\hspace{5mm}
\clearpage
\noindent {\huge{概要}}\\ \vspace{5mm}

{\large{
% TEX STUDIO MAGIC-COMMAND
% !TeX document-id = {21ffa6e2-6c8f-4532-897c-386dc477f19a}
% !TeX root = abstract.tex
%%% ↑ TeXのファイル名にあわせること.
% !TeX encoding = utf8
% !TeX TXS-program:compile = lualatex -synctex=1 -interaction=nonstopmode -halt-on-error %.tex
% !TeX TXS-program:quick = txs:///compile | txs:///view-pdf-internal --embedded

%%%-------------------------------------------------------------------------
%%% PD3予稿集テンプレート (main.tex)
%%% 作成: 金沢工大・情報工学科・鷹合研究室(2022,01/12)
%%%-------------------------------------------------------------------------

%%%%%%%%%%%%%%%%%%%%%%%%%%%%%%%%%%%%%%%%%%%%%%%%%%%%%%%%%%%%%%%%%%%%%%%%%%%
%                               テーマ,著者情報をここに書き込んでください
%ここから ------------------------------------------------------------------

%%% テーマ番号
\def\THEMEID{坂本研-04}

%%% タイトル
\def\TITLEJP{観光案内アプリにおけるPWAの活用に関する検討}
\def\TITLEEN{A Study on the Use of PWA in Tourist Information Apps}
\def\CENTERADJ{2.2} % ここを書き換えて,表紙の「プロジェクトテーマ」という文字列がセル中心になるよう調整してください

%%% 教員名
\def\PROFNAME{坂本 真仁 講師}

%%% アブストラクト(英文で書く)
% 最低:100ワード,最大:300ワード前後
% 英文部分については,句読点は半角にすること.つまり", "か". "を使う
\def\ABSTRACT{
With the spread of mobile devices such as smartphones, native mobile applications are now used by many people. Such native applications provide functions such as caching for faster loading and offline access. Developers can use these features to reduce data traffic and improve user experience. Progressive Web Apps(PWA) can be used to implement similar functionality in web applications. The advantages and challenges of PWA in tourist information applications will be evaluated by comparing them with native and regular web applications.
}

%%% キーワード(5個まで)
\def\KEYWORDS{Progressive Web Apps, Mobile}

%%% 著者リスト
\def\AUTHORS{
\begin{minipage}{13.5cm}
4EP1-25~笹川 尋翔(SASAGAWA Hiroto)
\end{minipage}
}

% テーマ,著者情報ここまで -----------------------------------------------------


%%%%%%%%%%%%%%%%%%%%%%%%%%%%%%%%%%%%%%%%%%%%%%%%%%%%%%%%%%%%%%%%%%%%%%%%%%%%
%                                本文
\documentclass{tkglabs}

\begin{document}
\maketitle
\begin{multicols*}{2} % *アスタ付きだとページのバランシングを無効にできる
%本文ここから ------------------------------------------------------------------

\section{背景}
情報端末の普及や多様化に伴い、特定のアプリプラットフォームに依存しないシステムであるWebが果たすべき役割が拡大している。この影響を受けて、現在はWHATWGやW3CによるWebの標準化が進められており、WebAssemblyによる処理の高速化やWeb APIによる高度な画像/動画処理が実現できるようになるなどの進展が見られる。モバイル端末において利用できるアプリとして、主にネイティブアプリケーションとWebアプリケーションがある。

ネイティブアプリケーションは一般的に、アプリストアと呼ばれる、ユーザーがアプリをダウンロードしたり購入したりするプラットフォームからダウンロードされ、インストールされる。ネイティブアプリケーションの長所は、オフラインアクセスが利用できる、カメラやプッシュ通知などのモバイル端末の機能にアクセスできる、適切な設計を行うことでWebアプリケーションよりも優れたユーザー体験を提供できる点である。

Webアプリケーションは、Webブラウザ上で動作するアプリである。Webアプリケーションの長所は、アプリのインストールが不要である、アプリのサイズがネイティブアプリケーションよりも小さい、更新がすぐに行われる点である。

2015年にProgressive Web Apps(PWA)という技術が登場した。これを活用することで、Webアプリケーションにおいてプッシュ通知、オフラインアクセス、ホーム画面へのロードなどの機能をネイティブアプリケーションと同じように利用できる。表~\ref{table1}に、PWAとネイティブアプリケーション、およびWebアプリケーションのそれぞれの特徴を示す。

\begin{table*}
  \centering
  \tabcap{モバイル端末で利用できる主なアプリの特徴}{Main app features available on mobile devices}{}
  \label{table1}
\begin{tabular}{|p{10em}|p{10em}|p{10em}|p{10em}|}
\hline
          & ネイティブアプリケーション & PWA    & 通常のWebアプリケーション      \\ \hline
インストール方法    & アプリストアにアクセス   & ``ホーム画面に追加する''ボタンを選択 & できない                \\ \hline
アップデート方法    & アプリストアにアクセス  & ページを再読み込み            & ページを再読み込み        \\ \hline
クライアント側で保持するファイルサイズ       & 大きい           & 小さい                     & 小さい                 \\ \hline
ユーザー体験    & 良い & 比較的良い     & 比較的悪い \\ \hline
プッシュ通知    & できる           & できる                     & サードパーティーのサービスを利用 \\ \hline
オフラインアクセス & できる           & できる                     & できない                \\ \hline
\end{tabular}
\end{table*}

モバイル端末が広く活用されるようになったことで、観光客の利便性向上を目的とした観光案内アプリも数多くリリースされている。日本においては、青森市\cite{AomoriCityTravelNavi}、 長野県\cite{ShinshuNavi}、和歌山市\cite{WakayamaSightseeing}などの様々な自治体が観光案内アプリを提供している。このようなアプリの特徴の1つとして、大量の画像、動画、音声などのコンテンツを扱うことが挙げられる。モバイル端末は、必ずしも高速で安定したネットワーク環境において使用されるとは限らないため、大量のコンテンツを効率良く扱うためにはキャッシュの管理が有効である。また、観光案内の役割を果たすために、プッシュ通知を通じて観光客の活動を促進することも重要である。これらの機能をPWAを用いて実装することで、開発工数を削減したり、Webアプリケーション開発の知識を活用したりできる。

\section{関連研究}
Tandelらの論文では、PWAの構成要素としてApp Shell、Service Worker、マニフェストがあることが示されている\cite{Tandel2018ProgressiveWebApps}。PWAでは、コンテンツから分離されたアプリのUIを``シェル''という。App ShellはシェルをキャッシュすることでWebアプリケーションをオフライン環境で快適に動作させるためのモデルである。Service Workerはバックグラウンドタスクを処理するためのエントリーポイントとして動作するスレッドを実行するモジュールである。マニフェストはアプリに関する情報を提供するファイルであり、アプリの名前、説明、作者、アイコンのパスをJSONテキストファイルに記述したものである。

Majchrzakらの論文では、クロスプラットフォーム開発が依然として重要なテーマであることを述べた上で、実例に基づいた研究によって、PWAの利便性がどの程度であるかを明らかにすることを今後の課題として挙げている\cite{Majchrzak2018ProgressiveWebApps}。2023年3月28日にリリースされたiOS 16.4でPWAのプッシュ通知に対応するなどの変更があったため、この研究では、最新の情報を元に改めてiOSにおけるPWAへの対応状況を説明する。

\section{目的}
観光案内アプリという実例に基づいて、PWAの利便性の程度を明らかにする。PWAは、ネイティブアプリケーションに近い動作をWebアプリで実現するものである。そこでこの研究では、PWAの利便性は、ネイティブアプリケーションで利用できる機能との互換性とPWAのパフォーマンスに基づくと解釈して、PWAを用いて実装した観光案内アプリの利便性を評価する。

\section{アプリの概要}
まず、観光案内アプリとして満たすべき機能を整理する。初めに必要とされる機能は、公共交通機関と観光名所の場所を示すことである。一般的に地図アプリを用いてこの機能を実装することが多いため、このアプリでは、Google MapsにGeoJSONデータをインポートしてそれらの場所を示す。GeoJSONとは、地理的データを表すJSONオブジェクトである。公共交通機関と観光名所のGeoJSONデータは、Overpass turbo\cite{OverpassTurbo}から取得する。Overpass turboは、Overpass APIを使用してOpenStreetMapのデータを取得できるWebサービスである。

地図に示された場所をユーザーが選択した際に詳細を表示する機能も必要である。詳細ページでは、JSONPlaceholder\cite{JSONPlaceholder}から取得した画像とテキストを表示する。JSONPlaceholderは、ダミーデータを提供するAPIである。作成するアプリの画面遷移図を図~\ref{figure1}に示す。

\begin{figure}
  \centering
  \includegraphics[width=\linewidth]{fig/screen_transition_diagram.png}
  \figcap{作成するアプリの画面遷移図}{Screen transition diagram of the application to be created}{}
  \label{figure1}
\end{figure}

\section{評価方法}
初めに、オフラインアクセス、キャッシュ、プッシュ通知などの機能の処理の流れを、ネイティブアプリケーション、PWA、通常のWebアプリケーションの間で比較する。次に、Lighthouse\cite{Lighthouse}というベンチマークツールを用いて、PWAと通常のWebアプリケーションのパフォーマンスの違いを計測する。Lighthouseでは、ネットワークやCPUの速度などをエミュレートしてパフォーマンスを計測したり、リクエストやレスポンスの内容を確認したりできる。
動作確認に用いる端末は以下の通りである。モバイル端末のOSごとにPWAの動作が異なるため、複数の端末を使用して動作を確認する。

\begin{itemize}
  \item motorola edge 20 fusion
    \begin{itemize}
      \item OS: Android 12
    \end{itemize}
  \item iPhone SE 第2世代
    \begin{itemize}
      \item OS: iOS 16.5
    \end{itemize}
\end{itemize}

\section{今後の展望}
実際の観光案内アプリにおいて、SDKを用いて実装されている機能を調査する。それと平行して、PWAを組み込むためのWebアプリケーションを作成する。作成したWebアプリケーションのパフォーマンスを適切に評価するために、Lighthouseで測定できる指標とその概要を調査することも必要である。また、Lighthouseではネットワークエミュレーションを使用した評価も行うため、基地局の性能や回線エリアのカバー率に関する最新の情報を調査して、エミュレートする回線速度を検討することも今後の課題である。

%% 参考文献(必要に応じて追加)
\begin{thebibliography}{9}
\bibitem{AomoriCityTravelNavi} 青森市, ``青森市観光ナビゲーションアプリ「青森市観光ナビ」-Aomori City Travel Navi-のご案内,'' \url{https://www.city.aomori.aomori.jp/kouryuu-suishin/kannkonabi.html} (2023年9月1日アクセス).
\bibitem{ShinshuNavi} 長野県, ``信州ナビ,'' \url{https://shinshu-navi.net/lp/} (2023年9月1日アクセス).
\bibitem{WakayamaSightseeing} 和歌山市, ``和歌山市観光アプリ,'' \url{http://www.city.wakayama.wakayama.jp/kankou/kankouspot/1027585/1027586.html} (2023年9月1日アクセス).
\bibitem{Tandel2018ProgressiveWebApps} S. S. Tandel and A. Jamadar, ``Impact of Progressive Web Apps on Web App Development,'' \textit{International Journal of Innovative Research in Science, Engineering and Technology}, vol. 7, pp. 9439-9444, Sep. 2018.
\bibitem{Majchrzak2018ProgressiveWebApps} T. A. Majchrzak \textit{et al.}, ``Progressive Web Apps: the Definite Approach to Cross-Platform Development?,'' in \textit{Proceedings of the 51st Hawaii International Conference on System Sciences}, IEEE Computer Society, Jan. 2018. pp. 5735-5744. 
\bibitem{OverpassTurbo} M. Raifer, ``overpass turbo,'' \url{https://overpass-turbo.eu/} (2023年9月1日アクセス).
\bibitem{JSONPlaceholder} typicode, ``JSONPlaceholder - Free Fake REST API,'' \url{https://jsonplaceholder.typicode.com/} (2023年9月11日アクセス).
\bibitem{Lighthouse} Chrome Developers, ``Lighthouse,'' \url{https://developer.chrome.com/docs/lighthouse/} (2023年9月1日アクセス).
\end{thebibliography}

% 本文ここまで ------------------------------------------------------------------
\end{multicols*}
\end{document}

\vfill
\noindent {\huge{実施報告}}\\ \vspace{5mm}

{\bfseries{活動履歴}}
\begin{center}
\begin{tabular}{rlr} \hline
  \multicolumn{1}{c}{ {\bfseries {期間} }} &
  \multicolumn{1}{c}{ {\bfseries {活動内容} }} &
  \multicolumn{1}{c}{ {\bfseries {活動時間[h]} }}
    %
% 参考
%
%% 期間 & 活動内容 & 活動時間[h] \\ % 未来の自分宛てメモ
%

\\ \hline
4月  & リサーチプロポーザルの見直し             & 80 \\ % 本当は就活していました
5月  & 文献調査                                 & 80 \\ % 本当は就活していました
6月  & 文献調査および内容提案                   & 80 \\ % ほとんどゲームしていました
7月  & 内容精査および文献調査                   & 80 \\ % 早めの夏休みをエンジョイしていました
8月  & プロトタイプ実装および中間報告会準備     & 80 \\ % ほとんどバイトをしていました.旅行にも行きました
9月  & 資料作成と中間報告                       & 10 \\ % まだ夏休みをエンジョイしていました
10月 & デバッグと改良                           & 80 \\ % 本格的に卒研に取り掛かり始めました
11月 & システムの評価                           & 80 \\ % わからないことだらけだったがなんとかなるだろうと思っていました
12月 & プロジェクトレポート執筆                 & 80 \\ % 終わらない気がしてきました
1月  & プロジェクトレポート執筆および公聴会準備 & 120 \\ % デバッグと発表資料・卒論作成に追われていました
2月  & プロジェクトレポート執筆・提出           & 30 % 論文執筆が間に合いました.なんとかなりました
\\ \hline


\end{tabular}
\end{center}
}}
\clearpage
\pagestyle{headings}
\pagenumbering{roman}
\setcounter{page}{1}
%%% 本文中では clearpageを使用してはならない
%%%%%%%%%%%%%%%%%%%%%%%%%%%%%%% ここまで表紙
\tableofcontents

% 以下の目次は不要であれば消す
% 図目次
\listoffigures
% 表目次
\listoftables
% ソースコード目次
\lstlistoflistings

\clearpage
\pagenumbering{arabic}
\setcounter{page}{1}

%% ここから本文を書く


\section{初めに}\label{section:初めに}
通信インフラの整備状況は地域間で大きく異なる。例えば、モバイルネットワークの世代別のシェアを見ると、世界では2022年時点で4Gネットワークが88\%を占めるが、アフリカでの4Gネットワークの普及率は50\%、アラブ諸国では76\%となっている~\cite{ITUFactsAndFigures2022}。所得別で見ると低中所得以上の人々の間では比較的4Gネットワークの普及率が高いが、低所得の人々の間の普及率は34\%と低い値を示している。その他には、都市と地方の間にも同様の格差が残っていることが指摘されている。これらのデータは何を示唆しているのだろうか?
\subsection{モバイルネットワークの普及率の格差をもたらす要因}\label{subsection:モバイルネットワークの普及率の格差をもたらす要因}
\subsubsection{通信インフラへの投資額の違い}\label{subsubsection:通信インフラへの投資}
通信インフラへの投資が、モバイル通信システムの新しい世代の普及率に影響を及ぼしていることが指摘されている~\cite{ForgeFormingA5GStrategyForDevelopingCountries2020}。通信インフラはモバイル通信サービスやオンラインサービスの基礎であるため、国々の間で激しい競争が行われているが、経済規模の違いから発展途上国や地方自治体は投資される側において不利である。膨大な予算を確保しやすい先進国や中央自治体の方がより魅力的な投資政策を立てられるためである。

他方で、通信システムを整備する組織の1つであるMNO(Mobile Network Operator)は、既にある通信設備の維持をしながら新しいモバイルネットワーク技術に多額の資金を投資する必要があるため、ある程度のリスクを抱えており、政府からの支援によって投資への積極性が左右されやすい。最近ではオンラインサービスを運営するMNOも登場しており、市場規模が拡大するにつれて明確な投資判断を下すために考慮するべき要素がますます増えている。
\subsubsection{モバイル通信技術の費用の増加}\label{subsubsection:モバイル通信技術の費用の増加}
モバイル通信システムの世代が新しくなるにつれ、モバイル通信技術を導入するための費用が増加する傾向にある。例えば、モバイル通信システムの世代が新しいほど使用する周波数帯が高くなる傾向にあるが、これは1つの基地局が対応できる通信範囲が狭くなることを意味する。これにより、特定のカバレッジを確保する場合にかかる基地局などの設備費用が増加する。

さらに、高速な通信規格においては、5Gの要件であるeMBB(enhanced Mobile Broadband)、mMTC(massive Machine Type Communication)、uRLLC(Ultra-Reliable and Low Latency Communications)のような厳しい基準が定められており、開発に求められる技術が複雑化している。これは人件費の増加を招くため、モバイル通信技術を導入する際の1つの障壁となっている。
\subsubsection{新しいモバイル通信技術の需要の不明確さ}\label{subsubsection:新しいモバイル通信技術の需要の不明確さ}
新しいモバイル通信技術はより高速なネットワークを提供するため、より多くの動画コンテンツに対応し、新しい市場や需要を産出する可能性があるが、その一方で現在の市場やエコシステムを大きく変える可能性もある。\autoref{subsubsection:通信インフラへの投資}で説明したように、発展途上国や地方自治体は先進国や中央自治体に比べて受け取る投資額が少ないため、需要がどのくらいあるのかが明確ではない、新しいモバイル通信技術の開発や導入に消極的である。
\subsection{PWAの概要}\label{subsection:PWAの概要}
\autoref{subsection:モバイルネットワークの普及率の格差をもたらす要因}で説明したように、通信技術の開発や通信インフラの整備状況は様々な要因の影響を受ける。これらの要因により十分な通信環境が確保できない人々に対しても、快適なユーザー体験を提供することが必要である。特に、モバイル端末は安価である一方で、利用時には移動体通信が行われるため通信環境が変化しやすく、快適なユーザー体験を維持することが難しい。このような問題を解決するために提案されている技術がPWA(Progressive Web Apps)である。

PWAはモバイルネイティブアプリに代表される、プラットフォーム固有のアプリのようなユーザー体験を提供するWebの技術である。通常のWebアプリでは柔軟なキャッシュ管理ができないため、Webページの読み込み速度をキャッシュレベルで改善するための仕組みが複雑である。また、常にオンラインでアクセスする必要があるため、通信環境の地理的な制約を受けることがある。PWAを活用することでこれらの問題を解決できる。PWAを構成する要素をいくつか示す。
\subsubsection{Web app manifest}\label{subsubsection:Web app manifest}

\subsubsection{Service Worker}\label{subsubsection:Service Worker}

\subsubsection{Web API}\label{subsubsection:Web API}

\section{関連研究}
\label{section:関連研究}
関連研究としては、PWAを構成する要素の性質を調査したもの、PWAとモバイルネイティブアプリのパフォーマンスの比較によってPWAの有用性を評価したもの、PWAの開発方法に着目し、クロスプラットフォーム開発における役割を整理したものなどがある。それぞれの概要を説明し、関連研究における課題を整理する。
\subsection{PWAの要素の性質}
\label{subsection:PWAの要素の性質}
PWAを構成する要素であるApp Shell、Service Worker、Web Application Manifestのそれぞれの性質は以下の通りである~\cite{Tandel2018ProgressiveWebApps}。
\subsubsection{App Shell}
\label{subsubsection:App Shell}
\begin{itemize}
    \item パフォーマンスが高い
    \begin{itemize}
      \item キャッシュされたコンテンツを利用することで高速にページを読み込めるため、通常のWebアプリと比べてパフォーマンスが高い。これはアプリに再びアクセスした際にページが即座に読み込まれることを意味する。
    \end{itemize}
    \item ネイティブアプリに近い操作性を実現
    \begin{itemize}
        \item Cache APIを利用することでオフライン環境でも動作する。PWAが考案される前は、ネイティブアプリのみがオフラインアクセスをサポートしていた。PWAによってネイティブアプリに近い操作性を実現するWebアプリが登場したことで、Webアプリの多様化が進んだ。
    \end{itemize}
    \item データの使用効率が高い
    \begin{itemize}
        \item UIをキャッシュするためページ遷移を行う際のデータ使用量が削減される。これによってデータの使用効率が高くなり、より少ないデータ使用量でページをレンダリングできるようになる。
    \end{itemize}
\end{itemize}
\subsubsection{Service Worker}
\label{subsubsection:Service Worker}
\begin{itemize}
    \item オフラインアクセスを提供
    \begin{itemize}
        \item キャッシュしたネットワークレスポンスを使用してオフラインアクセスを提供できる。通信が不安定な環境でもアプリを利用できるようになるため、ユーザビリティーが向上し、ユーザーに新たな価値を提供できるようになる、
    \end{itemize}
    \item プッシュ通知を提供
    \begin{itemize}
        \item ユーザーのモバイル端末やデスクトップ端末に直接送信されるメッセージであるプッシュ通知を提供できる。プッシュ通知を効果的に活用することで、アプリの利用を促進してコンバージョン率を向上させたり、行動パターンをより詳細に分析したりできるようになる。
    \end{itemize}
    \item バックグラウンドでコンテンツを更新
    \begin{itemize}
        \item Periodic Background Synchronization APIを用いることでバックグラウンドでコンテンツを更新できる。コンテンツを同期する際にユーザーの直接的な操作が不要になり、ユーザーに常に最新の情報を提供できる。
    \end{itemize}
    \item 柔軟なキャッシュ制御
    \begin{itemize}
        \item 一部のコンテンツのみをキャッシュしたり、一定の時間間隔でキャッシュを更新したりできるため、Cache Storageの使用量を削減してアプリのサイズを小さくしたり、古いコンテンツがキャッシュされたままになるのを回避できる。
    \end{itemize}
\end{itemize}
\subsubsection{Web Application Manifest}
\label{subsubsection:Web Application Manifest}
\begin{itemize}
    \item アプリに関する情報を提供
    \begin{itemize}
        \item アプリの名前、説明、作者、アイコンのパスなどの情報を提供できる。これらのメタデータをWebアプリに追加することで、Webアプリを特定のプラットフォームに適合させられる。
    \end{itemize}
\end{itemize}
\subsubsection{PWAの要素の性質に関する研究の課題}
\label{subsubsection:PWAの要素の性質に関する研究の課題}
実際のアプリでは、PWAが理想とする振る舞いを必ずしも実現できるとは限らない。例えば、バックグラウンドでのコンテンツの更新をサポートしている主なWebブラウザーは2023年11月時点でGoogle ChromeとMicrosoft Edgeのみであり、バックグラウンド同期に関する懸念も表明されている。また、Webブラウザー間でPWAの実装が異なるため、PWAの現状を分析する際は、PWAがそれぞれのWebブラウザーでどのように実装されているのかを考慮する必要がある。
\subsection{パフォーマンスの比較}
\label{subsection:パフォーマンスの比較}
Redditから画像とテキストを取得して表示する機能を、PWAとAndroidのネイティブアプリの両方に実装し、アプリの最初のアクティビティが起動するまでの時間や、アプリのアイコンをタップしてからツールバーがレンダリングされるまでの時間を測定した研究がある~\cite{Andreas2018ProgressiveWebApps}。レンダリング時間が高速であり、プラットフォームに準拠したAPIを使用しており、専有する容量が小さいというPWAの特徴を根拠として、クロスプラットフォーム分野でのPWAの競争力は高いと結論付けている。
\subsubsection{パフォーマンスの比較に関する研究の課題}
\label{subsubsection:パフォーマンスの比較に関する研究の課題}
アプリのパフォーマンスは複合的なものであり、アプリの起動が始まるまでの時間やレンダリング完了までの時間に加えて、アプリのナビゲーションのしやすさも検証するべきである。また、パフォーマンスをより正確に測定するためにはネットワークのスループット、端末の幅と高さ、端末の性能も考慮する必要がある。関連研究ではいずれも考慮されていないため、パフォーマンスを測定する方法の改善が必要である。
\subsection{クロスプラットフォーム開発における役割}
\label{subsection:クロスプラットフォーム開発における役割}
現在のクロスプラットフォームフレームワークはプラットフォーム間で技術を統一できない~\cite{Majchrzak2018ProgressiveWebApps}。関連研究では、この問題を解決するための方法の1つとしてPWAを挙げている。PWAは単一のコードで複数のプラットフォームに対応できる点で他のクロスプラットフォームフレームワークとは異なる。PWAのようなクロスプラットフォームフレームワークの登場により学習工数やコストが削減され、市場投入までの時間が短縮される。
\subsubsection{クロスプラットフォーム開発における役割に関する研究の課題}
\label{subsubsection:クロスプラットフォーム開発における役割に関する研究の課題}
プラットフォームごとにPWAのサポート状況が異なるため、PWAのWebブラウザー間の互換性を保つことが難しい。PWAの活用状況によってはPollyfillの導入が必要になる場合があり、それによってJavaScriptのサイズが大きくなり、クロスプラットフォーム開発の優位性が損なわれるかもしれない。
\subsection{PWAで活用できるWeb APIの調査}\label{subsection:PWAで活用できるWeb APIの調査}
ネイティブアプリの機能の多くはSDK (Software Development Kit)により提供されている。これは特定のフレームワークやプラットフォーム上にアプリケーションを構築するために使用する開発ツールのセットである。Webアプリ開発では、SDKの代わりにWeb APIを使用できる。API (Application Programming Interface)はプログラム同士が相互に通信するための方法である。SDKと同様に、開発者が複雑な機能をより簡単に作成できるようにするために提供されており、APIを使用することで複雑なコードが抽象化され、構文がより簡潔になる。Web APIはこのAPIの1つであり、HTTPなどのWebの技術を利用したものである。このように、Web APIはPWAをプラットフォーム固有のアプリ(ネイティブアプリ)に近づけるために不可欠な技術であるため、PWAで活用できるWeb APIを調査する。Web APIにはいくつかの種類があるが、この論文では特に、WebブラウザーのAPIのうち、W3Cなどにより標準化されたものを扱うことにする。以後この論文では、Webブラウザーの標準化されたAPIという意味でWeb APIという語句を使用する。
\subsubsection{Web APIのWebブラウザー側の対応状況の調査}
\label{subsubsection:Web APIのWebブラウザー側の対応状況の調査}
開発者が実際に特定のWeb APIを利用するためには、WebブラウザーがそのWeb APIに対応している必要がある。特に、シェアが大きいWebブラウザーの対応状況や、レンダリングエンジンが異なるWebブラウザー間の対応状況を把握することで、PWAの影響を受けるユーザーの範囲や割合を推測できる。主要なWebブラウザーのシェアや特徴は以下の通りである。なお、シェアはすべてのプラットフォームの2023年9月時点のデータを基に算出している~\cite{StatCounterBrowserMarketShare}。
\begin{itemize}
    \item Google Chrome
    \begin{itemize}
        \item シェア: 約63\%
        \item レンダリングエンジン
        \begin{itemize}
            \item HTML: Blink
            \item JavaScript: V8
        \end{itemize}
        \item チャンネル~\cite{GoogleChromeChannels}
        \begin{itemize}
            \item Stable
            \item Extended Stable
            \item Beta
            \item Dev
            \item Canary
        \end{itemize}
    \end{itemize}
    \item Safari
    \begin{itemize}
        \item シェア: 約20\%
        \item レンダリングエンジン
        \begin{itemize}
            \item HTML: WebKit
            \item JavaScript: Nitro
        \end{itemize}
        \item チャンネル~\cite{SafariChannels}
        \begin{itemize}
            \item Safari
            \item Beta
            \item Technology Preview
        \end{itemize}
    \end{itemize}
    \item Microsoft Edge
    \begin{itemize}
        \item シェア: 約5\%
        \item レンダリングエンジン
        \begin{itemize}
            \item HTML: Blink
            \item JavaScript: V8
        \end{itemize}
        \item チャンネル~\cite{MicrosoftEdgeChannels}
        \begin{itemize}
            \item Stable
            \item Extended Stable
            \item Beta
            \item Dev
            \item Canary
        \end{itemize}
    \end{itemize}
    \item Mozilla Firefox
    \begin{itemize}
        \item シェア: 約3\%
        \item レンダリングエンジン
        \begin{itemize}
            \item HTML: Gecko
            \item JavaScript: SpiderMonkey
        \end{itemize}
        \item チャンネル~\cite{MozillaFirefoxChannels}
        \begin{itemize}
            \item Firefox
            \item Extended Support Release(ESR)
            \item Beta
            \item Developer Edition
            \item Nightly
        \end{itemize}
    \end{itemize}
\end{itemize}
次に、前述した情報を踏まえ、Webブラウザー間におけるWeb APIの対応状況の違いを考える上で、着目するべき点を挙げる。まず、Google Chromeのシェアとその他のWebブラウザーのシェアの間に大きな差があることが分かる。次に、Google ChromeとMicrosoft Edgeの特徴が似ていることが分かる。Microsoft EdgeはGoogle Chromeと同様にChromiumというWebブラウザーから派生しているためである。Web APIの対応状況についてもほとんど同じである。そこで、Google ChromeとMicrosoft Edgeは同一のWebブラウザーであるとみなし、よりシェアが大きいGoogle Chromeを調査対象とする。

続いて、HTMLのレンダリングエンジンの違いに注目する。まず、SafariはBlinkのフォーク元であるWebKitを利用している。そのため、SafariにはGoogle ChromeやMicrosoft Edgeと共通のコードベースが含まれる可能性がある。さらに、WebKitは独自のエコシステムを持っている。例えば、iOS上のWebブラウザーはWebKit以外のHTMLレンダリングエンジンを使用できない。これにより、iOS上のWebブラウザー間の機能の違いが少なくなり、iOSにデフォルトでインストールされているSafariの市場優位性が高まる。Mozilla Firefoxについては、シェアは少ないものの、WebKitから完全に独立したHTMLレンダリングエンジンであるGeckoを採用している。また、プライバシーを重視する傾向にあり、PWAに関連するWeb APIの策定に積極的であるGoogle Chromeとは対照的である。そのため、Google Chromeに加えてMozilla Firefoxも調査対象とする。

次に、Web APIのWebブラウザー側の対応状況を調べる際に参照する文献の候補を示す。Can I use…~\cite{CanIUse}は様々なWebブラウザーがサポートする機能を検索できるWebサイトである。CC BY 4.0ライセンスを採用し、コミュニティーがWebサイトの情報を更新している。Can I use…に加えて、より信ぴょう性が高い情報を得るために、それぞれのWebブラウザーベンダーが提供しているRelease notesを参照する。Release notesは、新しいバージョンのソフトウェアをリリースする際に公表される、以前のバージョンからの変更点を示す文書である。また、より詳細な情報を入手するために、必要に応じてWebブラウザーベンダーのブログ記事を参照する。
\subsubsection{Web APIに対する意見や主張の調査}
\label{subsubsection:Web APIに対する意見や主張の調査}
それぞれのWeb APIに対する意見や主張は、Web APIの有用性を測るための1つの指標となり得る。Web APIに対する意見や主張を表明するための文書としてはStandards Positionsがある。Standards Positionsは、Web標準の技術に対する特定の開発コミュニティーの立場をまとめたものである。主な立場としては賛成 (support、positive)、中立 (neutral)、反対 (oppose、negative)がある。GitHubなどのプラットフォーム上で、特定のWeb標準の技術に対する賛否の議論が行われ、その後に開発コミュニティーの代表者が立場を表明する。まず、Standards Positionsでの特定の開発コミュニティーの立場と、Standards Positionsで表明された立場の根拠となる議論を調査し、Web APIの現状を考察する。さらに、Web APIに対する議論の方向性を基にWeb APIの将来の見通しを考察する。

メディアを制御するWeb APIのWebブラウザーの対応状況を表~\ref{table:メディアを制御するWeb APIの対応状況}に示す。そのWeb APIに対するベンダーの意見を表~\ref{table:メディアを制御するWeb APIに対する意見}に示す。
\begin{table}
  \caption{メディアを制御するWeb APIの対応状況}
  \label{table:メディアを制御するWeb APIの対応状況}
  \centering
  \begin{tabular}{|p{13em}|p{8em}|p{8em}|}
    \hline
    & Google Chrome & Mozilla Firefox \\ \hline
    Media Capture and Streams API & \cellcolor{gray!10}対応 & \cellcolor{gray!10}対応 \\ \hline
    MediaStream Recording API & \cellcolor{gray!10}対応 & \cellcolor{gray!10}対応 \\ \hline
    Media Session API & \cellcolor{gray!10}対応 & \cellcolor{gray!10}対応 \\ \hline
    Media Stream Image Capture API & \cellcolor{gray!10}対応 & \cellcolor{gray!10}対応 \\ \hline
    Media Capabilities API & \cellcolor{gray!10}対応 & \cellcolor{gray!10}対応 \\ \hline
  \end{tabular}
\end{table}
\begin{table}
  \caption{メディアを制御するWeb APIに対する意見}
  \label{table:メディアを制御するWeb APIに対する意見}
    \centering
    \begin{tabular}{|p{13em}|p{13em}|p{13em}|}
        \hline
        & Mozilla & WebKit \\ \hline
        Media Capture and Streams API & 不明 & \cellcolor{gray!10}OSの違いに柔軟に対処できる方法でOSがサポートする機能を公開できる\cite{WebKitMediaCaptureandStreamsAPI} \\ \hline
        MediaStream Recording API & 不明 & 不明 \\ \hline
        Media Session API & \cellcolor{gray!10}モバイル端末で特に有用である~\cite{MozillaMediaSessionAPI} & 不明 \\ \hline
        Media Stream Image Capture API & 不明 & 不明 \\ \hline
        Media Capabilities API & \cellcolor{gray!10}主要なWebサイトで実際に使用されている\cite{MozillaMediaCapabilitiesAPI} & 不明 \\ \hline
    \end{tabular}
\end{table}
Media Capture and Streams APIはカメラによるストリーミングを提供する。MediaStream Recording APIと組み合わせることで、音声や動画のストリーミングを収録できる。これらのAPIを活用すると、画面共有やビデオ通話といった機能を簡単に実装できる。Media Stream Image Capture APIと組み合わせることで写真を撮影することもできる。getUserMedia()によってカメラやマイクへの明示的なアクセス許可を求められるため、ユーザーが処理の流れを理解しやすい点や、再生ボタンをクリックするなどのアクションに関連付けられている点で挙動が明確である。

モジュールにアクセスするWeb APIのWebブラウザーの対応状況を表~\ref{table:モジュールにアクセスするWeb APIの対応状況}に示す。そのWeb APIに対するベンダーの意見を表~\ref{table:モジュールにアクセスするWeb APIに対する意見}に示す。
\begin{table}
  \caption{モジュールにアクセスするWeb APIの対応状況}
  \label{table:モジュールにアクセスするWeb APIの対応状況}
  \centering
  \begin{tabular}{|p{13em}|p{8em}|p{8em}|}
    \hline
    & Google Chrome & Mozilla Firefox \\ \hline
    Geolocaion API & \cellcolor{gray!10}対応 & \cellcolor{gray!10}対応 \\ \hline
    Web Bluetooth API & \cellcolor{gray!10}対応 & \cellcolor{gray!30}非対応 \\ \hline
    Web NFC API & \cellcolor{gray!30}非対応 & \cellcolor{gray!30}非対応 \\ \hline
    Vibration API & \cellcolor{gray!10}対応 & \cellcolor{gray!10}対応 \\ \hline
  \end{tabular}
\end{table}
\begin{table}
    \centering
    \caption{モジュールにアクセスするWeb APIに対する意見}
    \label{table:モジュールにアクセスするWeb APIに対する意見}
    \begin{tabular}{|p{13em}|p{13em}|p{13em}|}
         \hline
         & Mozilla & WebKit \\ \hline
         Geolocaion API & 不明 & 不明 \\ \hline
         Web Bluetooth API & \cellcolor{gray!30}セマンティックではないインターフェイスがWebプラットフォームに公開される~\cite{MozillaWebBluetoothAPI} & 不明 \\ \hline
         Web NFC API & \cellcolor{gray!30}物理的な認証デバイスのデータをWebサイトが取得する可能性がある~\cite{MozillaWebNFCAPI} & 不明 \\ \hline
         Vibration API & 不明 & \cellcolor{gray!30}通知の仕組みが悪用される可能性がある \\ \hline
    \end{tabular}
\end{table}
Geolocation APIはユーザーの位置情報を取得する。ユーザーの位置を地図上にプロットしたり、ユーザーの位置情報を用いて、パーソナライズされた情報を表示したい場合に便利である。プライバシー上の理由から明示的なアクセス許可やHTTPS通信が必要である。
Web Bluetooth APIはBluetooth Low Energyの周辺機器に接続して操作する機能を提供する。しかし、WebアプリがBluetooth端末に接続する場合に、そのアプリがどのような意図でその端末に接続しようとしているのかが明確に定義できないため、潜在的な危険性が高いと指摘されている。

バックグラウンド処理を行うWeb APIのWebブラウザーの対応状況を表~\ref{table:バックグラウンド処理を行うWeb APIの対応状況}に示す。そのWeb APIに対するベンダーの意見を表~\ref{table:バックグラウンド処理を行うWeb APIに対する意見}に示す。
\begin{table}
  \caption{バックグラウンド処理を行うWeb APIの対応状況}\label{table:バックグラウンド処理を行うWeb APIの対応状況}
  \centering
  \begin{tabular}{|p{13em}|p{8em}|p{8em}|}
    \hline
    & Google Chrome & Mozilla Firefox \\ \hline
    Background Tasks API & \cellcolor{gray!10}対応 & \cellcolor{gray!10}対応 \\ \hline
    Background Fetch API & \cellcolor{gray!10}対応 & \cellcolor{gray!30}非対応 \\ \hline
    Background Synchronization API & \cellcolor{gray!10}対応 & \cellcolor{gray!30}非対応 \\ \hline
  \end{tabular}
\end{table}
\begin{table}
  \caption{バックグラウンド処理を行うWeb APIに対する意見}
  \label{table:バックグラウンド処理を行うWeb APIに対する意見}
    \centering
    \begin{tabular}{|p{13em}|p{13em}|p{13em}|}
        \hline
        & Mozilla & WebKit \\ \hline
        Background Tasks API & 不明 & 不明 \\ \hline
        Background Fetch API & \cellcolor{gray!30}バックグラウンドでスクリプトが実行される~\cite{MozillaBackgroundFetchAPI} & \cellcolor{gray!30}ユーザーがWebサイトにアクセスしていないときにWebサイトのアクティビティーが実行される~\cite{WebKitBackgroundFetchAPI} \\ \hline
        Background Synchronization API & \cellcolor{gray!30}バッテリーが消耗しやすい、最初にアクセスしたネットワークと異なるネットワークでアクティビティーが発生する~\cite{MozillaBackgroundSynchronizationAPI} & \cellcolor{gray!30}電力とセキュリティーの懸念がある \\ \hline
    \end{tabular}
\end{table}
Background Tasks APIはタスクをキューに入れて優先度が高いものから順番に実行する。キューに入れられたタスクはバックグラウンドで実行されるため、Web Workerを使用せずにシステムの遅延を削減できる長所がある。Background Fetch APIは大容量のファイルをバックグラウンドでフェッチする。オフラインでフェッチリクエストが実行された場合は、ユーザーが再びオンラインになるまでプロセスを一時停止させることもできる。Background Fetch APIが実行する処理は実質的にはバックグラウンドでのダウンロードやアップロードであり、それらの処理の後にスクリプトがバックグラウンドで実行される可能性がある。ユーザーの明示的な操作を必須にすることや、Background Fetch APIがSame-Originのコンテンツのみを扱えるように制限するべきであるという声もある。Background Synchronization APIは安定したネットワーク接続が確立されるまでタスクを延期する。このAPIを使用して延期されたタスクは、異なるネットワーク上で実行される可能性があるため、ユーザーが意図しない処理が行われることが懸念されている。

ファイルにアクセスするWeb APIのWebブラウザーの対応状況を表~\ref{table:ファイルにアクセスするWeb APIの対応状況}に示す。そのWeb APIに対するベンダーの意見を表~\ref{table:ファイルにアクセスするWeb APIに対する意見}に示す。
\begin{table}
  \caption{ファイルにアクセスするWeb APIの対応状況}\label{table:ファイルにアクセスするWeb APIの対応状況}
  \centering
  \begin{tabular}{|p{13em}|p{8em}|p{8em}|}
    \hline
    & Google Chrome & Mozilla Firefox \\ \hline
    File API & \cellcolor{gray!10}対応 & \cellcolor{gray!10}対応 \\ \hline
    File System Access API & \cellcolor{gray!10}対応 & \cellcolor{gray!10}対応 \\ \hline
    File System API & \cellcolor{gray!10}対応 & \cellcolor{gray!10}対応 \\ \hline
  \end{tabular}
\end{table}
\begin{table}
  \caption{ファイルにアクセスするWeb APIに対する意見}
  \label{table:ファイルにアクセスするWeb APIに対する意見}
    \centering
    \begin{tabular}{|p{13em}|p{13em}|p{13em}|}
        \hline
        & Mozilla & WebKit \\ \hline
        File API & 不明 & 不明 \\ \hline
        File System Access API & \cellcolor{gray!30}APIのリスクをユーザーに適切に伝える方法が不明確である\cite{MozillaFileSystemAccessAPI} & \cellcolor{gray!30}ユーザーの利益を保護しながらローカルファイルシステムへの書き込みアクセスを許可する方法が不明確である~\cite{WebKitFileSystemAccessAPI} \\ \hline
        File System API & 不明 & 不明\\ \hline
    \end{tabular}
\end{table}
File APIは端末のファイルとそのコンテンツにアクセスする。File APIよりも汎用性が高いAPIとしてはFile System Access APIがある。File System Access APIは端末のファイルシステム上のファイルにアクセスしてファイルを読み込んだり、ファイルに書き込んだりできる。File System APIはWebブラウザーに仮想ドライブを作成し、そのストレージにファイルを保存するものであり、File System APIとFile System Access APIはそれぞれ異なるAPIである。File System Access APIは端末のすべてのディレクトリとファイルにアクセスする権限を持つため、悪用されるリスクがある。許可を求めるプロンプトの表示などによって、このAPIが行おうとしている操作をユーザーに分かりやすく提示することで安全性を確保できるという意見もある。

通知を制御するWeb APIのWebブラウザーの対応状況を表~\ref{table:通知を制御するWeb API}に示す。そのWeb APIに対するベンダーの意見を表~\ref{table:通知を制御するWeb APIに対する意見}に示す。
\begin{table}
  \caption{通知を制御するWeb API}
  \label{table:通知を制御するWeb API}
  \centering
  \begin{tabular}{|p{13em}|p{8em}|p{8em}|}
    \hline
    & Google Chrome & Mozilla Firefox \\ \hline
    Push API & \cellcolor{gray!10}対応 & \cellcolor{gray!10}対応 \\ \hline
    Notifications API & \cellcolor{gray!10}対応 & \cellcolor{gray!10}対応 \\ \hline
    Badging API & \cellcolor{gray!10}対応 & \cellcolor{gray!30}非対応 \\ \hline
  \end{tabular}
\end{table}
\begin{table}
  \caption{通知を制御するWeb APIに対する意見}
  \label{table:通知を制御するWeb APIに対する意見}
    \centering
    \begin{tabular}{|p{13em}|p{13em}|p{13em}|}
        \hline
        & Mozilla & WebKit \\ \hline
        Push API & 不明 & 不明 \\ \hline
        Notifications API & 不明 & 不明 \\ \hline
        Badging API & \cellcolor{gray!10}書き込み専用であるためプライバシーの観点で優れている~\cite{MozillaBadgingAPI} & 不明 \\ \hline
    \end{tabular}
\end{table}
Push APIはWebアプリがサーバーからプッシュ通知を受信する機能を提供する。Webアプリがフォアグラウンドで動作していなくても利用できるため、任意のイベントを通知してユーザーの関心を集められる。WebアプリからOSなどのシステムに対して通知を送るためにはNotifications APIを用いる。Badging APIを併用することで通知の数などの状態が変化したことをユーザーに通知できる。Notifications APIとBadging APIは、HTTPSでサーバーと通信しており、かつService WorkerなどのWebワーカーが動作しているWebアプリでのみ利用できる。

\section{研究内容}\label{sec:研究内容}

研究内容を述べる.

式~\eqref{eq:easy}に示すような複雑な数式であっても,\LaTeX{}なら簡単に美しく表現できる.
\begin{align}\label{eq:easy}
  \nabla^2&=\frac{1}{r^2}\pdv{r}\left(r^2\pdv{r}\right)+\frac{1}{r^2\sin\theta}\pdv{\theta}\left(\sin\theta\pdv{\theta}\right)+\frac{1}{r^2\sin^2\theta}\pdv[2]{\varphi}.
\end{align}

参考文献はこのように参照する~\cite{竹下隆史2012マスタリング}.


\section{評価}
\label{section:評価}
\subsection{Service Workerのパフォーマンス}
\label{subsubsection:Service Workerのパフォーマンス}
Lighthouseとネットワークスロットリングを使用して、それぞれの都道府県の観光地図を読み込んだ際のパフォーマンスを計測した。キャッシュ戦略ごとのパフォーマンス指標の値の平均を図~\ref{figure:Service Workerを使用しなかった場合のパフォーマンス}、~\ref{figure:全てのネットワークレスポンスをキャッシュした場合のパフォーマンス}、~\ref{figure:画像をキャッシュした場合のパフォーマンス}、~\ref{figure:Same-Originのネットワークレスポンスをキャッシュした場合のパフォーマンス}に示す。
\begin{figure}
  \centering
  \includegraphics[width=0.9\textwidth]{images/without_service_worker.png}
  \caption{Service Workerを使用しなかった場合のパフォーマンス}\label{figure:Service Workerを使用しなかった場合のパフォーマンス}
\end{figure}
\begin{table}
  \caption{全てのネットワークレスポンスをキャッシュした場合のパフォーマンス}
  \label{table:全てのネットワークレスポンスをキャッシュした場合のパフォーマンス}
  \centering
  \begin{tabular}{|p{15em}|p{5em}|p{5em}|p{5em}|}
    \hline
    & 3gslow & 3gfast & 4g \\ \hline
    パフォーマンス & 1 & 1 & 1 \\ \hline
    First Contentful Paint & 1 & 1 & 1 \\ \hline
    Largest Contentful Paint & 1 & 1 & 1 \\ \hline
    Speed Index & 1 & 1 & 1 \\ \hline
    Total Blocking Time & 1 & 1 & 1 \\ \hline
    Cumulative Layout Shift & 1 & 1 & 1 \\ \hline
  \end{tabular}
\end{table}
\begin{figure}
  \centering
  \includegraphics[width=0.9\textwidth]{images/service_worker_cache_images.png}
  \caption{画像をキャッシュした場合のパフォーマンス}\label{figure:画像をキャッシュした場合のパフォーマンス}
\end{figure}
\begin{figure}
  \centering
  \includegraphics[width=0.9\textwidth]{images/service_worker_cache_same_origin.png}
  \caption{Same-Originのネットワークレスポンスをキャッシュした場合のパフォーマンス}\label{figure:Same-Originのネットワークレスポンスをキャッシュした場合のパフォーマンス}
\end{figure}
いずれの場合もTBTとCLSは1である。4g、3gfast、3gslowの順でパフォーマンス指標の値が大きい傾向にある。全てのネットワークレスポンスをキャッシュした場合は3gslow、3gfast、4gのいずれの場合も全ての指標の値は1である。画像をキャッシュした場合はFCPが最も通信速度の低下の影響を受けやすく、SIが最もその影響を受けにくいことが分かる。逆に、Same-Originのネットワークレスポンスのキャッシュした場合はSIが最も通信速度の低下の影響を受けやすく、SCPが最もその影響を受けにくいことが示されている。

3gfastに比べてアップリンクがおよそ12倍、ダウンリンクがおよそ6倍である4gプロファイルを使用した場合であってもほとんどのパフォーマンス指標の値は1未満である。全体的に見ると、3gslowと3gfast間のパフォーマンス指標の値の増加率は3gfastと4g間の増加率よりも大きい。

\section{まとめ}\label{section:まとめ}
%\begin{itemize}
%  \item Web API
%  \begin{itemize}
%    \item モバイル端末へのアクセス
%    \begin{itemize}
%      \item モバイル端末の主要な機能にアクセスするためのWeb APIの対応状況や懸念点を述べる
%    \end{itemize}
%    \item バックグラウンドでの動作
%    \begin{itemize}
%      \item バックグラウンドでWebアプリを動作させるために必要なWeb APIの短所と課題を述べる
%    \end{itemize}
%  \end{itemize}
%  \item Service Worker
%  \begin{itemize}
%    \item キャッシュ制御
%    \begin{itemize}
%      \item Service Workerを利用することでキャッシュを柔軟に制御でき、古い世代のモバイル通信システムを使用した場合や一部のコンテンツのみをキャッシュした場合でもパフォーマンスの低下を抑えられることを述べる
%    \end{itemize}
%  \end{itemize}
%\end{itemize}
\subsection{観光案内アプリにおけるPWAの評価と課題}\label{subsection:観光案内アプリにおけるPWAの評価と課題}
観光案内アプリはコンテンツの配信を中心とするサービスである。現在配信されている観光案内アプリに実装されている機能の多くは先にユーザーからのアクションを必要とするため、Background Fetch APIやBackground Synchronization APIを使用したバックグラウンドでのコンテンツの取得や更新の需要はまだ小さいことが分かる。これらのWeb APIを使用したバックグラウンド処理はWebページの再読み込みの手間やコンテンツのダウンロードの手間を削減するが、ユーザーがその処理の流れを理解しにくいためプライバシー上の懸念がある。

一方で、現在配信されている観光案内アプリにおいては、GPSなどを使用した位置情報の取得が普及していることから、位置情報は観光地図などの機能を適切に動作させるために不可欠な仕組みであると言える。この仕組みをPWAに取り入れるためにはGeolocation APIを使用することが一般的である。アプリがGeolocation APIを使用してユーザーの位置情報にアクセスするためには、ユーザーがアプリに対してプロンプト経由で明示的にアクセスを許可する必要があり、かつユーザーにとってその流れが直感的であるため、観光案内アプリにおいて活用が期待される。

アプリの進化の歴史を踏まえると、アプリのインタラクティブ性は今後ますます高まると考えられる。前述したように、現在配信されている観光案内アプリの多くは受動的であり、アプリからユーザーへの積極的な情報発信が少ない。しかし、今後アプリのインタラクティブ化が進めば、ユーザーに適した観光名所の提案や、新しい観光名所の紹介がアプリのアクションをきっかけに行われるようになる可能性もある。このような機能がより一般的になれば、Push APIとNotifications APIを用いたプッシュ通知の価値が高まり、最終的にはPWAの有用性が向上するかもしれない。

ユーザー間で観光名所の情報を共有する機能を持った観光案内アプリもある。このような機能を実装する場合はMedia Capture and Streams APIを始めとするメディアの制御に関するWeb APIを利用できる。このWeb APIを用いることでファイルやストリームを容易に作成できる。ファイルやストリームを操作するWeb APIは多くのWebブラウザーがサポートしているが、ファイルシステムにアクセスするためのWeb APIは権限が大きすぎるという理由で一部のWebブラウザーベンダーが反対している。そのため、ディレクトリを管理するような複雑な操作をWebアプリに実装することは実質的には不可能である。

観光案内アプリにService Workerを追加するとコンテンツをキャッシュできるようになる。全てのネットワークレスポンスをキャッシュした場合は、2回目以降のページ読み込みではユーザーの端末内のストレージからそのコンテンツが取り出されるため、瞬時にページが表示され、最大のパフォーマンスを得られる。しかし、全てのネットワークレスポンスをキャッシュする方法を採用するとユーザーの端末内のストレージ容量を過剰に消費してしまう可能性がある。このような場合は特定の形式のファイルのみをキャッシュしたり、特定のオリジンのコンテンツのみをキャッシュしたりすることが推奨される。観光案内アプリで、画像をキャッシュした場合とSame-Originのネットワークレスポンスをキャッシュした場合を比べると、画像をキャッシュした場合はFCPの値が小さく、Same-Originのネットワークレスポンスをキャッシュした場合はSIの値が小さい。Same-OriginのネットワークレスポンスはHTML、CSS、JavaScript、JSONを中心に構成されるが、これらはいずれも画像に比べてファイルサイズが小さいため、Same-Originのネットワークレスポンスをキャッシュした場合はSIの値が小さくなりやすいと考えられる。一方で画像をキャッシュした場合は、ページの最初のコンテンツをレンダリングするために必要なHTMLをサーバーから取得する必要があるため、FCPが小さくなりやすいと考えられる。観光案内アプリでService Workerを使用することで、それぞれのアプリの特徴に合ったキャッシュ管理ができ、クライアントのストレージ容量の消費を節約しながらアプリのパフォーマンスを改善できる。

\section*{謝辞}\addcontentsline{toc}{section}{謝辞}
この研究を行うために坂本真仁講師や松井くにお教授からご指導いただきました。また、研究に関する議論にご参加いただいた研究室のメンバーに改めて感謝いたします。

\bibliographystyle{junsrt}
\bibliography{reference.bib}


\section*{付録}\addcontentsline{toc}{section}{付録}

研究遂行に際して開発したスクリプトや,解決に苦労した事項,作成したソースコード等,本編で述べるまでは無いが,後輩や未来の自分のために残しておきたい内容をここに書く.
その際必ず本文中で参照する.たとえば「\autoref{src:constructor-destructor}ではC++におけるコンストラクタとデストラクタの宣言と実装のための書式を示している.」のようにきちんと参照して述べる.

参考文献と付録は下限であるページ数25以上の条件にカウントしない.



\begin{figure}[!tb]
	\begin{lstlisting}[caption=コンストラクタとデストラクタの書式, label=src:constructor-destructor, basicstyle=\ttfamily\scriptsize,escapechar=@]
class クラス名{
  private:
    プライベート変数; // 他のクラスからはアクセス不可
    プライベート関数(); // 他のクラスからはアクセス不可
	public:
	  クラス名(); //デフォルトコンストラクタ
	  クラス名(引数の型); //コンストラクタ1
	  クラス名(引数の型, 引数の型); //コンストラクタ2
	  ~クラス名();//デストラクタ
};

クラス名::クラス名(){
	// デフォルトコンストラクタの実装
}

クラス名::クラス名(引数の型 引数){//コンストラクタ1
	// コンストラクタの実装
}

クラス名::クラス名(引数の型 引数, 引数の型 引数){//コンストラクタ2
	// コンストラクタの実装
}

クラス名::~クラス名(){
	// デストラクタの実装
}

int main(){
	クラス名 変数名;//デフォルトコンストラクタの呼び出し
	クラス名 変数名(引数);//コンストラクタ1の呼び出し
	クラス名 変数名(引数, 引数);//コンストラクタ2の呼び出し
	クラス名 変数名[10];//デフォルトコンストラクタの呼び出し(10回)
	return 0;//デストラクタがreturnの直前にそれぞれの変数で呼び出される(合計13回)
}
	\end{lstlisting}
\end{figure}



\end{document}
